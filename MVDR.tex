\documentclass[a4paper,12pt]{report}
\usepackage[utf8]{inputenc}
\usepackage[english, russian]{babel}
\usepackage{pscyr} % Нормальные шрифты
\begin{document}
	MVDR Beamforming - это метод обработки сигналов, в котором массив датчиков фокусируется на объекте получающем сигнал. Фиксируется угловое положение датчиков в массиве относительно объекта-получателя, для этого ведем вектор $\alpha$ содержащий фазовые сдвиги каждого датчик. В случае равномерного линейного размещения датчиков:
	\begin{center}
		$ \alpha = (1, e^{-j\zeta_0},..., e^{-j(M-1)\zeta_0})^T,$
	\end{center}
	где $(.)^T$ обозначает транспонирование, $M$ - число датчиков в массиве, $\zeta_0 = 2\pi f \frac{d}{c} sin(\theta)$, $f$ - частота дискретизации, $d$ - расстояние между датчиками в массиве, $\theta$ - азимутальный фиксированный угол между датчиком и объектом получателем, $c$ - скорость звука.
	
	Выходной сигнал $y$ задается линейной комбинацией данных c $M$ датчиков, $x$ является матрицей сигналов с датчиков антенной решетки и $w$ - вектор весов: 
	\begin{center}
			$y = w^Hx.$
	\end{center}
	где $(.)^H$ обозначает комплексное-сопряжение и транспонирование.
	 
  	Метод стремится минимизировать общую выходную мощность сигнала, сохраняя при этом усиление равным единице в фиксированном направлении $vs$ распространения сигнала, что приводит к задаче минимизации с ограничениями.
	
	Мощность $P$ выходящего сигнала:
	\begin{center}
		$P = E\{|y|^2\} = E\{|w^Hx|^2\} = E\{(w^Hx)(x^Hw)\} = w^HE\{xx^H\}w = w^HKw,$
	\end{center}
	где $K = E\{xx^H\}$ - ковариационная матрица.
	
	При минимизации этого уравнения, установив производную равной нулю, мы получим вектор $w$ содержащий только нули, по существу мощность минимизирована, однако, как упоминалось ранее, метод сохраняет усиление равным единице в направлении распространения сигнала. Определить данное ограничение можно как:
	\begin{center}
		$\alpha^Hw = 1.$
	\end{center}
    Получаем следующую задачу минимизации:
    \begin{center}
    	$\min_w \{P\} = \min_w \{w^HKw\}$
    	при условии $\alpha^Hw = 1.$
    \end{center}
	Для ее решения применяем метод Лагранжа, позволяющий минимизировать с учетом одного или нескольких ограничений. Включим ограничение в уравнение мощности выходящего сигнала:
	\begin{center}
		$P = w^HKw + \lambda(\alpha^Hw - 1),$
	\end{center}
	где $\lambda$ - множитель Лагранжа. 
	При задании производной от $dP/dw$ равной нулю, получим оптимальный вектор весов:
	\begin{center}
		$\frac{dP}{dw} = Kw + \alpha \lambda = 0,$
	\end{center}
	откуда следует
	\begin{center}
		$w = -K^{-1} \alpha \lambda.$
	\end{center}
	Домножив данное уравнение на $\alpha^H$ слева получим выражение для $\lambda$:
	\begin{center}
		$\lambda = -\frac{1}{\alpha^HK^{-1} \alpha}.$
	\end{center}
	В итоге получаем выражение для вектора весов:
	\begin{center}
		$w = \frac{K^{-1} \alpha}{\alpha^HK^{-1} \alpha}.$
	\end{center}  
\end{document}